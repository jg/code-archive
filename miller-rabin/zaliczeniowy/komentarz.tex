\documentclass{article}
\usepackage[T1]{fontenc}
\usepackage[latin2]{inputenc}
\usepackage[polish]{babel}
\title{Test pierwszo¶ci Millera-Rabina}
\author{Juliusz Gonera}
\makeatletter
\def\imod#1{\allowbreak\mkern10mu({\operator@font mod}\,\,#1)}
\makeatother

\begin{document}
\maketitle
\section{Test Fermata}
Je¶li dla liczby p zachodzi : 
\begin{eqnarray}
	b^{p-1}\equiv1 \imod{p}
\end{eqnarray}
Mówimy ¿e liczba p jest pseudopierwszą przy podstawie b. 
Z drugiej strony je¶li zachodzi :
\begin{eqnarray}
	b^{n-1}\not\equiv1 \imod{n}
\end{eqnarray}
Liczba n jest liczb± z³o¿on±. 
Niestety (1) jest spe³nione tak¿e dla niektórych liczb z³o¿onych przy pewnych podstawach (istnieje nawet klasa liczb które spe³niaj± (1) przy ka¿dej podstawie). Zatem ¿eby zmniejszyæ prawdopodobieñstwo pomy³ki w tzw te¶cie Millera-Rabina stosuje siê nastêpuj±ce usprawnienia : 
\begin{



\end{document}
